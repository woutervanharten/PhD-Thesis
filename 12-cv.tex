\hyphenation{Nij-megen Har-den-berg}
\chapter*{Curriculum Vitae}
\addcontentsline{toc}{chapter}{Curriculum Vitae}
\markboth{Curriculum Vitae}{Curriculum Vitae}
Wouter Gerrit van Harten was born on the $10^{\text{th}}$ of May in 1995 in Hardenberg, the Netherlands.
From a young age, he joined the local Scouting association, which has sparked a (hopefully) lifelong relationship with nature through regional, national, and international events.

After attending the (unfortunately recently closed) elementary school in Hoogengraven, Wouter completed his secondary education by obtaining a $\text{VWO}^{\text{extra}}$ diploma in 2013 at Vechtdal College in Hardenberg.
During this time, he discovered his keen eye for the exact sciences, and choosing tertiary education was not difficult.

In 2013, Wouter enrolled in a double Bachelor's program in \emph{Applied Mathematics} and \emph{Applied Physics} at the University of Twente, which culminated in 2017 with a double bachelor's degree.
After finishing the double Bachelor's program, Wouter enrolled in the MSc program \emph{Applied Mathematics} at the University of Twente to pursue the mathematical sciences further.
In this program, he took the Applied Analysis track, where he focused on (numerical) dynamical systems, resulting in a Master's thesis on Numerical continuation in neural field models.

Although the studies were interesting, he spent much of his spare time at the local Study association W.S.G.~Abacus, where Wouter volunteered for many years.
At W.S.G.~Abacus, Wouter served on the board as treasurer for a year and organized study tours to Switzerland and the United States.
Next to the study association, Wouter has been (and still is) an active member of the local student Scouting organization Radix Enschede.
Here, he has spent considerable time in good company and volunteered on the association's board, serving as both lease coordinator and treasurer for several years.

After completing his tertiary education in Twente, Wouter took the opportunity provided by dr.~Laura Scarabosio to pursue a doctoral degree in the Applied Analysis group at the Radboud University Nijmegen.
Under Laura's supervision, Wouter has studied efficient numerical methods for high-dimensional uncertainty quantification problems.
During his PhD~studies, Wouter took up an internship at Alliander under the supervision of dr.~Sander Rieken for three months, where he worked on the practical application of Bayesian filtering~\cite{vanharten2025}.
Other notable achievements include the first prize in the SIAM Scientific Computing Hackathon in 2023 and a poster prize at the 2024 edition of the Woudschoten conference organized by the Dutch-Flemish Scientific Computing Society.

Next to his studies at the Radboud, Wouter has spent many hours with the local and national Scouting movement in various volunteer positions throughout the years.
Currently, he serves as treasurer at Plusscoutkring Trapperskampen, modernizing the financial administration.