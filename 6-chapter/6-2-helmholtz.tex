We study the Truncated Exterior Dirichlet Problem (TEDP) of the Helmholtz equation in two dimensions, as introduced in Chapter~\ref{ch:introduction}.
For the reader's convenience, we reiterate the most relevant aspects here.

Denoting $D=\left\{ \x\in\mathbb{R}^2\,:\,r_{in}(\theta(\x)) < |\x| < r_{out} \right\}$ with $\theta(\x)$ the argument of the vector $\x$, we consider the variational problem: for $\y\in Y\coloneqq [-1, 1]^N$, find $u(\y) \in H^{1}_{\Gamma_{in}}(D)$ such that
\begin{align}
    \int_D A(\y) &\nabla u(\y) \cdot \nabla v \dx - k^2\int_{D}n(\y) u(\y) v \dx - i k\int_{\Gamma_{out}} u(\y) v\dx \nonumber \\ &=\int_{\Gamma_{out}} \left( \frac{\partial}{\partial \hat{\bm{n}}} -ik\right)u_{in} v\dx, \qquad
    \text{for all }v \in H^{1}_{\Gamma_{in}}(D)\label{eq:weakform},
\end{align}
where $u_{in}$ is the incoming wave, $n(\y; \cdot)$ is a heterogeneous, real-valued refractive index of the medium, and $A(\y; \cdot)$ a heterogeneous coefficient, taking values in $\mathbb{R}^{2\times 2}$.
Moreover, we require both $n(\y,\cdot)$ and $A(\y,\cdot)$ to be $C^1$ in $\y$ to go beyond affine expansions in $\y$.
The central scatterer is impenetrable and corresponds to the region $\left\{ \x \in \mathbb{R}^2\,\|\, \|\x\| < r_{in}(\theta) \right\}$.


We consider a discretization of~\eqref{eq:weakform} with the finite element method, leading to linear systems of the form
\begin{equation}
    \mathbb{A}(\y)\u(\y)=b\label{eq:linear_system},
\end{equation}
where we assume that problem~\eqref{eq:weakform} is nontrapping.
To make this explicit, we denote
\begin{equation*}
    \left\| \cdot \right\|_{H_{k}^1(D)}^2\coloneqq \left\| \nabla \cdot \right\|_{L^2(D)}^2+k^2\left\| \cdot\right\|_{L^2(D)}^2
\end{equation*}
and formulate the nontrapping assumption:
\begin{assumption}[Nontrapping]\label{ass:nontrapping}
The quantities $D$, $A$, and $n$ are such that, given $f\in L^2(D)$, the solution $u(\y)$ to~\eqref{eq:weakform} exists, is unique, and satisfies
\begin{align*}
    \|u(\y)\|_{H_{k}^1(D)}\leq C_{\text{bound}} \|f\|_{L^2(D)},
\end{align*}
where $C_{bound}$ is independent of $k$ and $\y\in Y$.
\end{assumption}

For high $k$, solving~\eqref{eq:linear_system} becomes challenging with GMRES\@.
To solve it many times for different parameter values, we thus follow Chapter~\ref{ch:distributed-preconditioning} to assign the linear systems to a limited number of preconditioners using a surrogate of the number of GMRES iterations.
To this aim, we need a good approximation for the prior mean, which we will construct in the next section, and a measure of dimension importance in the parameter space, which we will discuss in Section~\ref{subsec:importance-weights}.

\subsection{The prior mean}\label{subsec:prior-mean}
Choosing a prior mean is a delicate task.
We could use the estimates in~\cite{graham2021} and set
\begin{equation}
    \mu_0(\y,C)\coloneqq C_1 \|A(\y; \cdot) - A(\hat{\y};\cdot)\|_{L^2_{op}(D)}+C_2 \|n(\y; \cdot) - n(\hat{\y};\cdot)\|_{L^2(D)}\label{eq:GP-mean-def},
\end{equation}
where $C_1$ and $C_2$ are hyperparameters and $\|\cdot\|_{L^2_{op}(D)}$ is the $L^2$-norm of the pointwise spectral matrix 2-norm.
However, computing this requires integration over $D$, which is costly.

To evaluate~\eqref{eq:GP-mean-def} efficiently, we exploit the differentiability of $A$ and $n$:
\begin{equation*}
    n(\y, \cdot) = n(\yh, \cdot) + \nabla_{\y} n(\bm{\eta}^1, \cdot) \cdot \left( \y - \yh \right),
\end{equation*}
for some $\bm{\eta}^1\in Y$ between $\y$ and $\yh$.
Expanding the second term in~\eqref{eq:GP-mean-def}, we obtain:
\begin{align*}
    \|n(\y; \cdot) - n(\hat{\y};\cdot)\|_{L^2(D)}&= \| \nabla_{\y} n(\bm{\eta}^1, \cdot) \cdot \left( \y - \yh \right) \|_{L^2(D)}\\
    &= \left( \left(\y - \yh \right)^\top\mathbb{B}(\bm{\eta}^1)\left(\y - \yh \right) \right)^{\frac{1}{2}}\\
    &= \left\| \y-\yh \right\|_{\mathbb{B}(\bm{\eta}^1)},
\end{align*}
where $\left\| \cdot \right\|_{\mathbb{B}(\bm{\eta}^1)}$ denotes a \emph{weighted $\ell^2$-norm} with a positive definite wei\-ght matrix $\mathbb{B}(\bm{\eta}^1)$, parameterized by $\bm{\eta}^1$.
In this case, $\mathbb{B}(\bm{\eta}^1)$ is a symmetric matrix with entries
\begin{equation}
    \mathbb{B}_{ij}(\bm{\eta}^1) = \int_D \left|  \partial^{e_i} n(\bm{\eta}^1, \cdot)\partial^{e_j} n(\bm{\eta}^1, \cdot)\right|  \dx \label{eq:Bmatdef}.
\end{equation}
Similarly, we obtain $\|A(\y; \cdot) - A(\hat{\y};\cdot)\|_{L^2_{op}(D)} = \left\| \y-\yh \right\|_{\mathbb{D}(\bm{\eta}^2)},$ where $\mathbb{D}(\bm{\eta}^2)$ is a symmetric matrix with entries
\begin{equation}
    \mathbb{D}_{ij}(\bm{\eta}^2) = \int_D\left|  \partial^{e_i} A(\bm{\eta}^2, \cdot)\partial^{e_j} A(\bm{\eta}{^2}, \cdot)\right|_{2,2}  \dx \label{eq:Dmatdef},
\end{equation}
where $\bm{\eta}^2\in Y$ lies between $\y$ and $\yh$ and $\left\| \cdot \right\|_{2,2}$ is the spectral matrix norm.

The matrices $\mathbb{B}(\cdot)$ and $\mathbb{D}(\cdot)$ can often be approximated by constant matrices by using the structure of the parameter dependence.
We will discuss this for an affine expansion in Section~\ref{subsec:disjoint-density} and a parameterized domain in Section~\ref{subsec:parameterized-domain}.
However, the results presented in both sections are more widely applicable than these two examples, which merely show the process of applying these methods.
We also note that because of the hyperparameters $C_1$ and $C_2$, it is sufficient to estimate the norms up to an (unknown) constant.

We can then combine~\eqref{eq:GP-mean-def} with $\mathbb{B}$ and $\mathbb{D}$ to obtain:
\begin{equation}
    \mu_0(\y|C)= C_1 \left\| \y-\yh \right\|_{\mathbb{D}}+C_2  \left\| \y-\yh \right\|_{\mathbb{B}}\label{eq:GP-mean-def_y},
\end{equation}
for the prior mean of the gray-box Gaussian process~\eqref{eq:alpha_GP_def}.

\subsection{Anisotropy weights}\label{subsec:importance-weights}
Next to the prior mean, we prescribe values for the anisotropy weights $\gamma_i$ proportional to the importance of dimensions in the parameter space.
These weights allow us to further exploit the structure in the parameter domain such that we can scale better to higher parameter dimensions.
Therefore, we define them in terms of the matrices $\mathbb{B}$ and $\mathbb{D}$:
\begin{equation*}
    \gamma_i \coloneqq C_1\sqrt{\mathbb{D}_{ii}}+C_2\sqrt{\mathbb{B}_{ii}}, \qquad i = 1,\ldots, N,
\end{equation*}
where $C_1$ and $C_2$ are the hyperparameters from equation~\eqref{eq:GP-mean-def}.
Now, the correlation length $l_i$ of the Matérn kernel in dimension $i$ is given by:
\begin{equation*}
    l_i = \text{diam}(D) \frac{\max_j \gamma_j}{\gamma_i} \geq \text{diam}(D),
\end{equation*}
where $\text{diam}(D)$ is the diameter of the computational domain.

In the next section, Section~\ref{subsec:disjoint-density}, we will consider an affine dependence on the parameter $\y$ for $n$, and in Section~\ref{subsec:parameterized-domain}, we consider a parameterized domain test case, where the shape of the central impenetrable scatterer depends on the parameter $\y$.

\subsection{Affine expansion for the refractive index}\label{subsec:disjoint-density}
We investigate affine dependence on the parameter $\y$ for $n$, and we consider a constant $A=I_2$.
The affine expansion of the refractive index $n$ is given by
\begin{equation}
    n(\y,\x) \coloneqq n_{0}(\x) + \sum_{i=1}^{N} \psi_i(\x) y_i, \label{eq:affinendef}
\end{equation}
with $n(\y, \x)$ uniformly positive with respect to $\x$ and $\y$, and such that the resulting problem is non-trapping for all $\y\in Y$.

To apply the methods we developed in Chapter~\ref{ch:distributed-preconditioning}, we need to estimate the matrices $\mathbb{D}$ and $\mathbb{B}$.
Since $A$ is independent of $\y$, we set $\mathbb{D}_{ij}=0$ for $i,j=1,\ldots, N$.
To estimate $\mathbb{B}$, we use equation~\eqref{eq:Bmatdef} and compute
\begin{equation}
    \mathbb{B}_{ij} =\int_D \psi_i(\x)\psi_j(\x) \dx,\qquad i,j=1,\ldots, N.\label{eq:affineBdef}
\end{equation}
In the numerical experiments, we will consider the expansion
\begin{equation}
    n(\y,\x) \coloneqq 1 + \sum_{i=1}^{N} \mathbbm{1}_{\Omega_i}(\x)\chi(\x)\eta_i \frac{y_i-1}{2}, \label{eq:ndef}
\end{equation}
for a partition $\{\Omega_i\}_{i=1}^N$ of $D$, weights $\eta_i\in\mathbb{R}$ with $\y\in Y$, and $\chi$ a mollifier satisfying the assumption below:
\begin{assumption}\label{ass:mollifier}
The mollifier $\chi(\x)$ is continuous on $D$, $\chi(\x)=1$ on $\Gamma_{in}$ and $\chi(\x)=0$ in an open neighborhood of $\Gamma_{out}$.
\end{assumption}

By employing a mollifier, we ensure that $n(\cdot, \x)\equiv 1$ and $\nabla_{\x} n(\y,\x)=0$ on the outer boundary $\Gamma_{out}$ such that we can apply the Robin approximation on $\Gamma_{out}$
Moreover, we choose the partition $\{\Omega_i\}_{i=1}^N$ of $D$ as
\begin{equation*}
    \Omega_i = \left\{\x \in D \,,\, \frac{2\pi i}{N} \leq \theta(\x) < \frac{2\pi (i+1)}{N} \right\},
\end{equation*}
such that, together with the term $(y_i-1)$ in~\eqref{eq:ndef}, the non-trapping Assumption~\ref{ass:nontrapping} is fulfilled~\cite[Condition~2.6]{graham2019}.
Moreover, $\mathbb{B}$ is a diagonal matrix with entries $\mathbb{B}_{ii}=C_\chi\eta_i^2$ where $C_\chi \in \mathbb{R}$ is a constant depending on the mollifier that will be absorbed into the hyperparameters.
This reduces the correlation lengths to
\begin{equation*}
    l_i = \diam(D) \frac{\max_j \eta_j}{\eta_i}, \qquad i=1,\ldots, N.
\end{equation*}
For a generic $n$, the matrix $\mathbb{B}$ from~\eqref{eq:affineBdef} can be computed similarly to the techniques we present in the next section.

\subsection{Parameterized domain}\label{subsec:parameterized-domain}
Our second test case considers scattering against a star-shaped object $D_{scat}(\y)$, parameterized by $\y\in Y = \left[ -1,1 \right]^N$ as introduced in Chapter~\ref{ch:uncertainty-quantification-for-paramterized-domains}.
For the reader's convenience, we remind the reader that $D$ corresponds to the reference configuration on which we have the variational problem: for every $\y\in Y$, find $u(\y) \in H^1_{0, \Gamma_{scat}}(D)$ such that
\begin{align*}
    \int_{D}& A(\y,\cdot) \nabla u(\y) \cdot \nabla v \dx - k^2\int_{D} n(\y,\cdot) u(\y) v \dx - i k\int_{\Gamma_{out}}  u(\y) v\dx \\
    &= \int_{\Gamma_{out}} \left( \frac{\partial}{\partial \hat{\bm{n}}} -ik\right)u_{in} v\dx +  \int_{D} f v\dx, \nonumber
    \text{\qquad for all }v \in H^1_{0, \Gamma_{scat}}(D),
\end{align*}
where the coefficients are
\begin{equation*}
    A(\y; \x) =  \D\Phi^{-1}(\y; \x) \D\Phi^{-\top}(\y; \x)=\det(\D\Phi(\y; \x)),
\end{equation*}
and
\begin{equation*}
    n(\y; \x) = \det(\D\Phi(\y; \x),
\end{equation*}
with $\Phi$ the mapping to the physical domain and $u$ is the pulled back solution.

Similar to the affine expansion in Section~\ref{subsec:disjoint-density}, we still have to obtain the constant matrices $\mathbb{B}$ and $\mathbb{D}$ by approximating $\eqref{eq:Bmatdef}$ and $\eqref{eq:Dmatdef}$.
To do this, we will use upper bounds in~\cite{harbrecht2016} on the first derivative of $A$ and $n$ with respect to $\y$.

We treat the $\mathbb{B}$ matrix first, by employing~\cite[Lemma~4]{harbrecht2016} to obtain:
\begin{equation*}
    \left|  \partial^{e_i}n(\bm{\eta}^1, \cdot)\right| \leq 2 (1+\sigma_{max})^2 \left\| \Phi_i \right\|_{W^{1,\infty}(D)}, \qquad i=1,\ldots, N,
\end{equation*}
for all $\bm{\eta}^1$, where $\sigma_{\max}$ is the upper bound from the Courant-Fisher Theorem, and $\Phi_i$ are the partial transformations from equation~\eqref{eq:phidefpois}.
Hence, we bound:
\begin{align*}
    \mathbb{B}_{ij}(\bm{\eta}^1) &= \int_D \left| \partial^{e_i} n(\bm{\eta}^1, \cdot)\partial^{e_j} n(\bm{\eta}^1, \cdot)   \right| \dx\\
    & \leq C_{\mathbb{B}} \left\| \Phi_i \right\|_{W^{1,\infty}(D)}  \left\| \Phi_j \right\|_{W^{1,\infty}(D)}\\
    & \eqqcolon C_{\mathbb{B}} \mathbb{B}_{ij},
\end{align*}
where the constant $C_\mathbb{B}=|D|4 (1+\sigma_{\max})^4$ will be absorbed into the hyperparameter $C_2$.
Similarly, we use ~\cite[Theorem~4]{harbrecht2016} to bound~\eqref{eq:Dmatdef}:
\begin{equation*}
    \mathbb{D}_{ij}(\bm{\eta}^2) \leq C_{\mathbb{D}} \mathbb{D}_{ij},
\end{equation*}
where
\begin{equation*}
    \mathbb{D}_{ij}=\mathbb{B}_{ij}=\left\|\Phi_i \right\|_{W^{1,\infty}(D)}  \left\| \Phi_j \right\|_{W^{1,\infty}(D)},
\end{equation*}
and
\begin{equation*}
    C_\mathbb{D}=\frac{16(1+\sigma_{\max})^2 }{\sigma_{\min}^2} \frac{4 (1 + c_\gamma)^2}{\sigma_{\min}^4\ln\left( 2 \right)^2}
\end{equation*}
is a constant which will be absorbed into the hyperparameter $C_1$.
Moreover, $c_\gamma=\sum_{j=1}^N \left\| \Phi_j \right\|_{W^{1,\infty}(D)}$.
In the numerical experiments, we will consider the Fourier expansion~\eqref{eq:fourierdef}, together with the mollifier mapping~\eqref{eq:exterior_mollifier}.
For these, we compute:
\begin{equation}
    \left\| \Phi_j \right\|_{W^{1,\infty}(D)} \leq
    \begin{cases}
        2\vartheta\left\| \nabla \chi(\x) \right\|_{L^{\infty}} ,& \text{for }j=1,\\
        \left( \frac{j+2}{2} \right)^{-\alpha}\vartheta\left( 1 + \left\| \nabla \chi(\x) \right\|_{L^{\infty}} + \left( \frac{j}{2} \right)  \right) , & \text{for $j$ even},\\
        \left( \frac{j+1}{2} \right)^{-\alpha}\vartheta\left( 1 + \left\| \nabla \chi(\x) \right\|_{L^{\infty}} + \left( \frac{j-1}{2} \right) \right) ,& \text{else}.\\
    \end{cases}\label{eq:shapew1infnormdef}
\end{equation}
Finally, we can use the estimates of $\mathbb{B}$ and $\mathbb{D}$ to compute the correlation lengths:
\begin{equation*}
    l_i = \diam(D) \frac{\max_j \left\| \Phi_j \right\|_{W^{1,\infty}(D)}}{\left\| \Phi_i \right\|_{W^{1,\infty}(D)}}, \qquad i=1,\ldots, N.
\end{equation*}

