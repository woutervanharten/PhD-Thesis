Uncertainty is a part of almost everything we do and experience.
It shows up in mundane ways, such as not knowing when a parcel will arrive or how high the next electricity bill will be.
However, it also plays a role in much higher-stakes situations, such as predicting the effects of climate change or ensuring the safety of a nuclear reactor.
Uncertainty is not always a problem, as we often seek it out in entertainment when anticipating the end of a movie or playing the lottery.
However, to make informed decisions in situations where uncertainty matters, one must understand the origins and effects of these uncertainties.
Therefore, the field of \emph{Uncertainty Quantification} (\emph{UQ}) studies the interplay between uncertainty and computational models.

On the one hand, we study the effects of uncertainty in \emph{input parameters} of computational models on their output.
For example, we might not know exactly how much rain will fall in the upcoming days, but we want to know the probability of a devastating flood.
Alternatively, we have an appointment at 09:00 in the morning, but we are unsure how much traffic there will be on our way to the office.
In these cases, we understand the underlying uncertainty, but we want to know the effect of this uncertainty on some \emph{Quantity of Interest} (QoI).
In the mentioned examples, the quantities of interest are the flood probability and the expected arrival time.

On the other hand, we may have uncertainty in latent variables that we cannot observe directly; we can only observe a consequence.
For example, we observe a novel disease spreading without knowing its infectivity or resistance against common countermeasures.
As we have experienced in the early '20s, it is critical to quantify these uncertainties using techniques from \emph{inverse uncertainty quantification} and \emph{filtering}.

Both aspects have a basic element in common: the computational (forward) model, which maps the input parameters to a quantity of interest or an observable.
For the forward UQ problem, the workflow is shown schematically in Figure~\ref{fig:uq_workflow_tikz}.
In the morning traffic example, $Y$ represents the traffic density on the road.
The forward model $\mathfrak{f}:Y\to X$ maps the traffic density to an arrival time, and $QoI(X) = \mathbb{E}\left[ X \right]$ computes the average arrival time.
In the flooding example, on the other hand, $Y$ could represent the rainfall, the geometry of the river bed, or other unknowns.
The forward model $\mathfrak{f}:Y\to X$ then computes the expected water height at a location of interest.
The quantity of interest would be $QoI(X) = \mathbb{P}\left[ X > \text{dike height} \right]$, the probability that the water will flow over the dike.

\begin{figure}
    \centering
    \begin{tikzpicture}
        \node[bn] at (0, 0) (a) {\begin{minipage}{3cm}\centering Uncertain input parameters \\$\mathcal{Y}$ \end{minipage}};
        \node[on] at (4, 0) (b) {\begin{minipage}{3cm}
                                     \centering
                                     Computational model\\
                                     $X=\mathfrak{f}(\mathcal{Y})$
        \end{minipage}};
        \node[rn] at (8, 0) (c) {\begin{minipage}{3cm}
                                     \centering
                                     Quantity of Interest\\
                                     $QoI(X)$
        \end{minipage}};
        \draw[to] (a) -- (b);
        \draw[to] (b) -- (c);
    \end{tikzpicture}
    \caption{Schematic representation of the UQ workflow.}
    \label{fig:uq_workflow_tikz}
\end{figure}

Many difficulties in this computation arise whenever the number of input uncertainties is large or possibly infinite, as in the case of unknown geometries or other spatially dependent parameters.
Difficulties arising from large parameter dimensions are collectively referred to as the \emph{curse of dimensionality} and pose a significant computational challenge.
Moreover, solving the forward model is often rather expensive, as we will discuss in this thesis.