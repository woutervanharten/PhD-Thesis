Next to the mollifier mapping discussed in the previous section, we introduce the harmonic extension~\cite{li2001,xiu2006} to map any Lipschitz domain $D$ to ${D}(\y)$.
This map is defined implicitly through one Laplace equation per spatial dimension.
For spatial dimension $d=2$, it is defined as the solution to:
\begin{align}
    \begin{cases}
        \Delta \Phi^\kappa = 0, &\text{ in }D,\\
        \Phi^\kappa = \phi^\kappa(\theta),  &\text{ on }\partial D,
    \end{cases} \label{eq:harmonicext}
\end{align}
where $\phi^\kappa(\theta) \coloneqq f_\kappa(\theta) r(\y; \theta)$ for $\kappa \in \{x, y\}$ and $f_x(\cdot) = \cos(\cdot), f_y(\cdot)=\sin(\cdot)$.
By linearity of the Laplace equation, we can conclude that, for $r(\y;\theta)$ given by~\eqref{eq:rdef}, the mapping $\Phi$ satisfies the decomposition~\eqref{eq:phidefpois}.
Moreover, we can conclude that each partial mapping $\Phi_j=[\Phi_j^x, \Phi_j^y]^\top$ solves the harmonic equation with $\Phi_j^\kappa = f_\kappa(\theta) \psi_j(\theta)$ as Dirichlet boundary condition, with $\phi_j^\kappa$ defined accordingly.

The main advantage of this approach is that it results in mappings with great smoothness properties.
The oscillatory behavior of the boundary expansion will be resolved near the boundary, removing the need for fine meshing at the center of the domain.
These improvements come at a cost, as we have to solve two partial differential equations for each mapping.
However, in practical applications, we make use of the linearity in~\eqref{eq:phidefpois} to precompute the partial transformations a priori and assemble each transformation separately, limiting the computational burden.
Moreover, by defining the mappings implicitly through an auxiliary equation, we gain difficulty in the theoretical analysis of the mapping.

Instead of the Laplace equation, we could consider different linear partial differential equations to define the mapping.
High numerical accuracy has been obtained with the elasticity equations~\cite{cizmas2008,dwight2009}, which offer even more control over the smoothing properties of the transformation by utilizing piecewise constant material properties~\cite{stein2003}.
However, rigorous theoretical analysis of the mapping will become even more complex when the equations become more involved.