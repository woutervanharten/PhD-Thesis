We parameterize the boundary $r(\y;\theta)$ with vectors $\y$ from the (possibly infinite-dimensional) hypercube $Y\coloneqq[-1, 1]^N$ with $N\in\mathbb{N}\cup \left\{ \infty \right\}$.
We model $r(\y;\cdot)$ with an affine combination of the nominal radius $r_0:[0,2\pi)\to\mathbb{R}_+$ and basis functions $\{\psi_j(\x)\}_{j \geq 1}$, by setting
\begin{align}
    r(\y;\theta) = r_0(\theta) + \sum_{j \geq 1}\psi_j(\theta)y_j,\label{eq:rdef}
\end{align}
for all $\theta\in[0, 2\pi)$ and all $\y\in Y$.
To ensure the required regularity of the domain mapping in Section~\ref{sec:mapping-approach}, we formulate the following assumption and consecutive lemma:
\begin{assumption} \label{ass:phiregularity}
For all $j \geq 1$, $\psi_j \in C^{0, 1}_{per}([0,2\pi))$, and there exists $\gamma \in (0,1)$, independent of $\y\in Y$, such that
\begin{equation*}
    \sum_{j \geq 1}|\psi_j(\theta)| \leq \gamma r_0(\theta),
\end{equation*}
and
\begin{equation*}
    \sum_{j \geq 1}|\psi_j'(\theta)|  < \infty,
\end{equation*}
for all $\theta\in[0, 2\pi)$.
Moreover, we have
\begin{equation*}
    r_0(\theta) \in C^{0,1}_{per}([0,2\pi )]).
\end{equation*}
\end{assumption}
\begin{lemma}
    By Assumption~\ref{ass:phiregularity}, we have
    \begin{equation*}
        r(\y; \cdot) \in C^{0, 1}_{per}([0,2\pi)),\quad \text{for all } \y \in Y,
    \end{equation*}
    and $\|r(\y, \cdot)\|_{C^{0,1}_{per}}$ has a $\y$-independent upper bound.
\end{lemma}
Moreover, Assumption~\ref{ass:phiregularity} ensures that $r(\y, \theta) > 0$ for all $\y\in Y$ and $\theta\in[0,2\pi)$.
This implies that the boundary of the scatterer is non-intersecting.

\subsubsection{Fourier expansion}
Several choices for the basis $\left\{ \psi_j(\theta) \right\}_{j\geq 1}$ can be considered.
Because the boundary is periodic, it is natural to explore the Fourier basis:
\begin{equation}
    \psi_j(\theta)\coloneqq\begin{cases}
                               \Theta, & \text{for }j=1,\\
                               \Theta \left(  \frac{j+2}{2}\right)^{-\alpha}\sin(j \theta/2 ), & \text{for $j$ even},\\
                               \Theta \left( \frac{j+1}{2} \right)^{-\alpha}\cos((j-1) \theta / 2 ), & \text{else},\\
    \end{cases}\label{eq:fourierdef}
\end{equation}
where
\begin{equation*}
    \Theta=\frac{\vartheta r_0^-}{1+\sqrt{2}\left( \zeta(\alpha) + 1\right)}.
\end{equation*}
Moreover, $0<\vartheta<1$ is a scaling constant determining the magnitude of the shape variations relative to the minimal unperturbed radius $r_0^{\min}=\min_{\theta\in [0,2\pi)}r_{0}(\theta)$, $\alpha>2$ a parameter that describes the decay in the Fourier modes, and $\zeta(\alpha)$ is the Riemann Zeta function evaluated at $\alpha$.

Now we have, by direct computation,
\begin{equation*}
    \sum_j | \psi_j(\theta) | \leq r_0^-\vartheta,
\end{equation*}
and
\begin{equation*}
    \sum_j | \psi_j'(\theta) | \leq \frac{r_0^-\vartheta\sqrt {2}\left( \xi(\alpha - 1) - 1 \right)}{1+\sqrt {2}\left( \xi(\alpha - 1) - 1 \right)}<\infty,
\end{equation*}
for a.e. $\theta\in[0,2\pi)$, such that the bounds in Assumption~\ref{ass:phiregularity} are satisfied.

\subsubsection{Wavelet expansion}
Next to the globally supported Fourier basis, we take a look at the special case where $\left\{ \psi_j \right\}_{j\geq 1}$ forms a wavelet basis.
Wavelet expansions allow for localized perturbations in $r(\y, \theta)$, in contrast to the globally supported Fourier modes, and may thus be attractive from a modeling point of view.
For an overview of wavelets, we refer to~\cite{meyer1994}.

Let $\{\psi_j(\theta)\}_{j\geq 1}$ be a frame of wavelets generated by periodization of a mother wavelet $\Psi \in C^{0, 1}(\mathbb{R})$ with $\|\Psi\|_{L^\infty(\mathbb{R})}\coloneqq1$, as introduced by Yves Meyer~\cite{meyer1994}.
As it is more natural, we denote the wavelets by their index $\lambda$, which is a concatenation of their space and scale levels, and we use the notation $\abs{\lambda}=l\geq 0$ for the scale level and $[\lambda]$ for the number within the scale level.
At each given scale level $l \geq 0$, there are $\mathcal{O}(2^{l})$ wavelets and their supports are such that we can define
\begin{equation}
    M\coloneqq\sup_{\theta \in [0,1)} \sum_{k \in \mathbb{Z}}\abs{\Psi(\theta - k)}, \label{eq:Mbound}
\end{equation}
and
\begin{equation}
    M_d\coloneqq\sup_{\theta \in [0,1)} \sum_{k \in \mathbb{Z}}\abs{\Psi'(\theta - k)}, \label{eq:Mbound3}
\end{equation}
for some $M, M_d \in \mathbb{R}$ independent of the level.
We note that the rapid decay of the mother wavelet ensures the finiteness of $M$ and $M_d$~\cite{meyer1994}.
More explicitly, we define, for all $\theta\in[0, 2\pi)$ and $\alpha>2$,
\begin{align}
    \psi_\lambda(\theta)&\coloneqq \frac{r_0^-\vartheta}{M}(1-2^{-\alpha}) 2^{-\alpha \abs{\lambda}} \sum_{k\in\mathbb{Z}}\Psi\left(2^{|\lambda|}\left(\frac{\theta}{2\pi}-k\right)-[\lambda]\right), \label{eq:waveletdef}
\end{align}
where again $0 < \vartheta < 1 $ is the overall maximal amount of shape variation relative to the minimal unperturbed radius $r_0^{\min}$.
The sum $\sum_{k\in\mathbb{Z}}$ is a direct effect of the periodization to the unit interval~\cite{meyer1992}.

Wavelets are such that, due to their localized support, their pointwise sum is well-defined, as expressed by the following lemma:
\begin{lemma}[Pointwise summability of wavelets] \label{lem:poitwisesum}
For a Lipschitz-continuous mother wavelet and $\alpha>2$ in~\eqref{eq:waveletdef}, we have the following pointwise bound on the sum of the wavelets and their derivatives:
\begin{align*}
    \sum_\lambda | \psi_\lambda(\theta) | &\leq r_0^-\vartheta,
\end{align*}
and
\begin{align*}
    \sum_\lambda | \psi_\lambda'(\theta ) | &\leq r_0^-\vartheta \frac{M_d}{M}\frac{\left( 1-2^{-\alpha}\right)}{\left( 1-2^{-(\alpha-1)}\right)},
\end{align*}
for a.e. $\theta \in [0,2\pi)$.
In particular, the radius expansion fulfills the bounds in Assumption~\ref{ass:phiregularity}.
\end{lemma}
\begin{proof}
    Let $\lambda_l\in \{ \lambda : |\lambda|=l \}$, $l\geq 0$.
    Using equations~\eqref{eq:waveletdef} and~\eqref{eq:Mbound}, we expand the first sum and obtain
    \begin{align}
        \sum_\lambda | \psi_\lambda(\theta) |
        \leq r_0^-\vartheta(1-2^{-\alpha}) \sum_{l\geq 0} 2^{-\alpha l}
        \leq r_0^-\vartheta. \label{eq:pointwiseone}
    \end{align}
    Similarly, expanding the second sum yields
    \begin{align}
        \sum_\lambda | \psi_\lambda'(\theta) |
        &\leq r_0^-\vartheta(1-2^{-\alpha}) \frac{M_d}{M} \sum_{l\geq 0} 2^{-(\alpha-1) l}\nonumber\\
        &\leq r_0^-\vartheta \frac{M_d}{M}\frac{\left(1-2^{-\alpha}\right)}{\left(  1-2^{-(\alpha-1)}\right)}.\label{eq:pointwisetwo}
    \end{align}
\end{proof}

The results presented here hold for wavelets in general, and in practical applications, we will have to choose a mother wavelet $\Psi(\cdot)$.
Common choices include compactly supported wavelets, such as the spline wavelets~\cite{unser1997}.
However, localized wavelets, like the orthogonal Daubechies family~\cite{daubechies1988}, can also be used with great effect.