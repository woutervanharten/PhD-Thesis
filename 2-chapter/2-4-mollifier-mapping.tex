
In this section, we follow~\cite{hiptmair2018} and examine a transformation that maps the nominal domain ${D}$ to the parameterized domain $\Dy$ by employing a mollifier $\chi(\x)$.
The main idea is to propagate the rescaling of the boundary radially to the interior of the domain using a mollifier $\chi(\x)$.
The main advantage of this approach is its analytic tractability, as we can compute derivatives explicitly, allowing for sharper asymptotic results.

\begin{assumption}\label{ass:mollifier_t}
The mollifier $\chi(\x):{D} \to [0, 1]$, $\chi \in C^{0, 1} (\overline{{D}})$, has the following properties:
\begin{enumerate}
    \item[(i)] For $|\x| < r_0(\theta(\x))$, it is monotonically increasing as a function of $|\x|$ for fixed $\theta(\x)$.
    \item[(ii)] For $|\x| > r_0(\theta(\x))$, it is monotonically decreasing as a function of $|\x|$ for fixed $\theta(\x)$.
    \item[(iii)] It is zero in a neighborhood of the origin in a neighborhood of non-parameterized boundaries and one on the parameterized boundary.
\end{enumerate}
\end{assumption}

\begin{definition}[Mollifier transformation]
    With a mollifier satisfying Assumption~\ref{ass:mollifier_t}, we define a mollifer transformation, mapping the nominal domain ${D}$ to the parameterized domain ${D}(\y)$, for every $\y\in Y$, by:
    \begin{equation}
        \Phi(\y;\x) = \x + \chi\left(\x\right)\left[ r(\y;\theta(\x)) - r_0(\theta(\x)) \right]\frac{\x}{|\x|},\quad  \x \in D. \label{eq:mollifiertransf}
    \end{equation}
\end{definition}

From Assumption~\ref{ass:mollifier_t}, we conclude that $\Phi(\y;\cdot)$ is bijective.
Also, it fits the framework~\eqref{eq:phidefpois} with, for $j\geq 1$,
\begin{equation}
    \Phi_j(\x)\coloneqq\chi\left(\x\right)\psi_j(\theta(\x))\frac{\x}{|\x|},\quad \x\in D.\label{eq:frameworkmapping}
\end{equation}
The exact definition of the mollifier is domain-specific, and we will discuss the mollifier mapping for both the interior domain and the truncated exterior domain in the upcoming sections.

\subsubsection{Interior domain}\label{subsubsec:interior-domain}
On the interior domain used for the Poisson equation, we define an affine mollifier by:
\begin{definition}[Interior affine mollifier]\label{def:interior_mollifier}
We define the affine mollifier on the interior domain by
\begin{equation}
    \chi(\x) = \begin{cases}\frac{|\x| - r_{mol}}{r_0(\theta(\x))- r_{mol}},&\text{for }|\x|\geq r_{mol},\\
    0, & \text{otherwise},
    \end{cases}
    \label{eq:interior_mollifier}
\end{equation}
where $r_{mol} < (1-\gamma )r_0^-$ with $r_0^-= \min_{\theta\in [0, 2\pi)} \left(r_0(\theta) \right)$ equal to the minimal nominal radius and $\gamma$ as defined in Assumption~\ref{ass:phiregularity}.
\end{definition}
This choice of $r_{mol}$ ensures the fulfillment of Assumption~\ref{ass:mollifier_t}.
The cutoff value $(1-\gamma )r_0^-$ has been chosen such that all mappable shape deformations can be accommodated.
However, for large values of $\gamma$, any oscillatory behavior of the basis functions $\psi_j$ gets amplified by a factor $\frac{1}{|\x|}$ near the origin.
This, in turn, increases the required resolution of the used mesh around the origin.
Hence, we have to choose the regularity of our basis functions such that we can choose $\gamma$ large enough not to impact the computational load too much.

\subsubsection{Truncated exterior domain}\label{subsubsec:truncated-exterior-domain}
On the external domain, the mollifier is defined similarly as
\begin{definition}[Exterior affine mollifier]\label{def:exterior_mollifier}
We define the affine mollifier on the exterior domain by
\begin{equation}
    \chi(\x) = \begin{cases}\frac{|\x| - r_{mol}}{r_0(\theta(\x))- r_{mol}},&\text{for }|\x|\leq r_{mol},\\
    0, & \text{otherwise},
    \end{cases}
    \label{eq:exterior_mollifier}
\end{equation}
where $r_{mol} \in ( r_0^++\gamma r_0^{-}, r_{out})$ with $r_0^-= \min_{\theta\in [0, 2\pi)} \left(r_0(\theta) \right)$ equal to the minimal nominal radius, $r_0^+= \max_{\theta\in [0, 2\pi)} \left(r_0(\theta) \right)$ equal to the maximal nominal radius, and $\gamma$ as defined in Assumption~\ref{ass:phiregularity}.
\end{definition}
Again, we note that the upper and lower bound on $r_{mol}$ ensure Assumption~\ref{ass:mollifier_t}.
Moreover, for this interval to be nonempty, we have the lower bound $r_{out} > r_0^++\gamma r_0^{-}$.
As we are dealing with an external domain, we are not limited by the oscillation concentration issues at the center of the internal domain.