\label{sec:modelproblem}
To show the relevance of Theorem~\ref{thm:lpsummability}, we consider the Poisson equation posed on the interior parameterized domain as introduced in Chapter~\ref{ch:uncertainty-quantification-for-paramterized-domains}, with the domain mappings introduced in Sections~\ref{subsec:mollifier-mapping} and~\ref{subsec:harmonic-mapping}.
We thus focus on approximating the map $\y\to u(\y)$ from $Y=[-1,1]^\mathbb{N}$ to $V=H_0^1(D)$, where $u(\y)$ is the solution to equations~\eqref{eq:transformedvarformpois} --~\eqref{eq:hatfdefpois}.

To efficiently compute evaluations of ${u}(\y)$, we are interested in a polynomial approximation for the mapping $\y \mapsto {u}(\y)$ from $Y$ to $V$.
The holomorphy of an extension of the map $\y \mapsto {u}(\y)$  from $Y$ to $V$ to complex poly-ellipses, under Assumptions~\ref{ass:analytic} and~\ref{ass:phijpois}, was shown in~\cite{chkifa2015,cohen2018,hiptmair2018}.
Therefore, we can expand the solution to~\eqref{eq:transformedvarformpois} in terms of a Taylor series with respect to the parameter~\eqref{eq:tayloreq}.
We will investigate the summability properties of the Taylor coefficients by employing Theorem~\ref{thm:lpsummability} in the following section.

\subsection{Setup for Theorem~\ref{thm:lpsummability}}
\label{subsec:verificationthm}
By Assumptions~\ref{ass:analytic},~\ref{ass:phijpois}, and Lemma~\ref{lem:courantfish2pois}, Assumption~\ref{ass:afanalytic} is satisfied.
To apply Theorem~\ref{thm:lpsummability}, we then need a sequence $(b_j(\x))_{j \geq 1}$ that verifies its assumptions.
In this section, we will first show   \textsl{pointwise} bounds on $|{A}(\bm{0}; \x)|_{2,2}$ and $|F(\bm{0}; \x)|$ from~\eqref{eq:hatAdefpois} and~\eqref{eq:hatfdefpois} in terms of $ |\D \Phi_j(\bm{0}; \x) |_{2,2}$ and $ | \Phi_j (\bm{0}; \x)|$, $j\geq 1$, and use these bounds to choose $(b_j(\x))_{j \geq 1}$ appropriately.

We define the following sequences for the bounds on the spectral norms of the total derivatives of $a$ and $f$, making use of the notation introduced at the beginning of Chapter~\ref{ch:exploiting-locality-in-polynomial-approximations}:
\begin{align*}
    \Ca(n)\coloneqq\sup_{\x \in \mathcal{D}_H}|\D^n a|_{n,2}, \quad \Cf(n)\coloneqq\sup_{\x \in \mathcal{D}_H}|\D^n f|_{n,2}.
\end{align*}
Moreover, we let $\Cat$ and $\Cft$ be the monotonically non-decreasing majorants of $\Ca$ and $\Cf$\rev{,} respectively.

Now, the desired bounds on $|A(\bm{0}; \x)|_{2,2}$ and $|F(\bm{0}; \x)|$ follow from direct computation and are given by the following lemma:
\begin{lemma}
    \label{thm:Afbound}
    For $\y \in Y $, let ${A}(\y)$ and $F(\y)$ be defined by equations~\eqref{eq:hatAdefpois} and~\eqref{eq:hatfdefpois}, respectively, and let Assumption~\ref{ass:analytic} and~\ref{ass:phijpois} hold.
    Then, for a.e. $\x \in {D}$ and for every $\mu \in \mathcal{F}$, we have the bounds
    \begin{align*}
        |\partial^\mu{A}(\bm{0}; \x)|_{2,2} &\leq \Cat(|\mu|)  \sum_{ \nu \leq \mu} \binom{\mu}{\nu} \frac{(|\mu| - |\nu| + 2)!}{2} \nonumber\\
        &\qquad((|\D \Phi_j(\bm{0}; \x)|_{2,2} )_{j \geq 1})^{\mu - \nu} \sqrt{d}^{\,|\mu - \nu|} ((| \Phi_j(\bm{0}; \x)| )_{j \geq 1})^{ \nu},
    \end{align*}
    and
    \begin{align*}
        |\partial^\mu F(\bm{0}; \x)| &\leq \Cft(|\mu|)   \sum_{\nu \leq \mu} \binom{\mu}{ \nu } |\nu |!\sqrt{d}^{|\nu |}\\
        &\qquad\qquad((|\D \Phi_j(\bm{0}; \x)|_{2,2})_{j\geq 1})^\nu   ((|\Phi_j(\bm{0}; \x)|)_{j \geq 1})^{\mu - \nu}.
    \end{align*}
\end{lemma}


Analogous results to Lemma~\ref{thm:Afbound} can be found, for example, in~\cite{harbrecht2016}, which employs the Faà di Bruno formula.
However, contrary to these results, we employed a different proof technique, and we only need to evaluate the derivatives at $\y=\bm{0}$, which results in our bounds being slightly sharper than those in~\cite{harbrecht2016}.
\begin{proof}
    We expand the derivative  $\partial^{\mu}{A}(\bm{y})$ using the general Leibniz rule and substitute $\y = \bm{0}$ to obtain:
    \begin{align}
        |\partial^\mu {A}(\bm{0}) |_{2,2}  &\leq \sum_{\nui+\nuii+\nuiii+\nuiv=\mu} \binom{\mu}{\nu^{(1)},\nu^{(2)},\nu^{(3)}, \nuiv}\,\,\,\, \cdot \nonumber\\
        &\qquad \qquad\left| \partial^{\nu^{(1)}}\left[ \left( \D \Phi \right)^{-1} (\bm{0})\right]\right|_{2,2} \left|   \partial^{\nu^{(2)}}\left[ \left( \D \Phi \right)^{-\top} (\bm{0})\right]   \right|_{2,2}\cdot\nonumber\\
        &\qquad \qquad\left|\partial^{\nu^{(3)}}\left[\det\left( \D \Phi \right) (\bm{0})\right]\right|_{2,2} | \partial^{\nuiv}\left[\left( a \circ \Phi \right)(\bm{0}) \right]|.\label{eq:inbetweenbound}
    \end{align}
    This leaves us with the task of estimating each norm on the right-hand side of the previous equation.

    First, we tackle $\left| \partial^\mu \left[ \left( \D \Phi \right)^{-1}(\bm{0})\right]\right|_{2,2}$, by computing the derivative explicitly.
%    We do so instead of employing the Fa\`a di Bruno formula to exploit the affine parameter dependence.
    We obtain:
    \begin{align*}
        \partial^\mu \left[ \left( \D \Phi \right)^{-1}(\bm{y})\right] &= (\D \Phi)^{-1}(\y) \sum_{\xi\in P(\mu)} \prod_{j=1}^{|\mu|}\left[ \D \Phi_{\xi_j} (\D \Phi)^{-1}(\y) \right],
    \end{align*}
    where $P(\mu)$ is the set of all possible permutations of the derivatives in $\mu$, each counted with its multiplicity.
    We can evaluate this at $\bm{y}=\bm{0}$, where $\D\Phi(\bm{0})=I_d$,  to obtain:
    \begin{align}
        \left| \partial^\mu \left[ \left( \D \Phi \right)^{-1}(\bm{0})\right]\right|_{2,2} &\leq  \sum_{\xi\in P(\mu)} \prod_{j=1}^{|\mu|} |\D \Phi_{\xi_j} |_{2,2}\\
        &=  \abs{\mu}! ((|\D \Phi_j |_{2,2})_{j \geq 1}) ^{\mu}.\label{eq:invbound}
    \end{align}
    Similarly, for the transpose, we obtain:
    \begin{align}
        \left| \partial^\mu \left[ \left( \D \Phi \right)^{-\top}(\bm{0})\right]\right|_{2,2} &\leq  \abs{\mu}! \left((|\D \Phi_j |_{2,2})_{j \geq 1}\right) ^{\mu},\label{eq:invtranspbound}
    \end{align}
    because the spectral norm is transpose invariant.
    Next, we take a look at the determinant by evaluating the derivative iteratively.
    For an arbitrary $j \in \text{supp}(\mu)$:
    \begin{align*}
        g_\mu^{\bm{y}}&\coloneqq\left| \partial^\mu \left[ \det\left( \D \Phi(\y) \right)\right]\right| \\
        &\phantom{:}=\left| \partial^{\mu - e_j}\left[ \det(\D \Phi(\y)) \tr\left((\D\Phi(\y)) ^{-1}\D\Phi_j\right)\right]\right|\\
        &\phantom{:}=\Bigg| \sum_{\nu \leq \mu - e_j}  \binom{\mu - e_j }{ \nu}  \partial^{\nu} \left[ \det\left( \D \Phi(\y) \right)\right] \\&\qquad\qquad\qquad\qquad\qquad\qquad\tr\left( \partial^{\mu - e_j - \nu}\left[ (\D \Phi(\y))^{-1} \right] \D \Phi_j  \right)\Bigg|\\
        &\phantom{:}\leq \sum_{\nu \leq \mu - e_j}  \binom{\mu - e_j }{\nu}  g_\nu^{\bm{y}} | \partial^{\mu - e_j - \nu}\left[ (\D \Phi(\y))^{-1} \right]|_F | \D \Phi_j |_F,
    \end{align*}
    with $|\cdot|_F$ the Frobenius norm.
    Evaluating at $\bm{y}=\bm{0}$ and employing~\eqref{eq:invbound}, we obtain
    \begin{align*}
        \left| \partial^\mu \left[ \det\left( \D \Phi(\bm{0}) \right)\right]\right| &= g_\mu^{\bm{0}}\\
        &\hspace{-30pt}\leq \sum_{\nu \leq \mu - e_j}  \binom{\mu - e_j }{ \nu}  g_\nu^{\bm{0}} (|\mu - \nu|-1)!((| \D \Phi_j |_F)_{j\geq 1})^{\mu - \nu},
    \end{align*}
    and we claim this recurrence relation is solved by
    \begin{equation}
        g_\mu^{\bm{0}} \leq \abs{\mu}!(|\D \Phi_j|_F)^\mu, \label{eq:detbound}
    \end{equation}
    for every $\mu\in\mathcal{F}$.
    To show this, we proceed by induction on $\abs{\mu}$.
    For $\mu = \bm{0}$, we have  $g_{\bm{0}}^{\bm{0}} = 1=|\bm{0}|!(|\D \Phi_j|_F)^{\bm{0}},$ where we used the convention that $0!=1$.
    Assuming~\eqref{eq:detbound} holds for $\abs{\mu} \leq n$, for $g_{\mu+e_j}^{\bm{0}}$, and any $j\geq 1$, we have:
    \begin{align*}
        g_{\mu+e_j}^{\bm{0}}
        &\leq((| \D \Phi_j |_F)_{j\geq 1})^{\mu}\sum_{\nu \leq \mu}  \binom{\mu }{ \nu}  \abs{\nu}! |\mu - \nu|!
        \\&=   (\abs{\mu}+1)!((| \D \Phi_j |_F)_{j\geq 1})^{\mu},
    \end{align*}
    where, in the last inequality, we used the case $N=2$ of Lemma~\ref{lem:multinomialsum} in Appendix~\ref{ap:combilemma}.

    Finally, we bound $| \partial^\mu (a \circ \Phi) |$ by exploiting the affine dependence of $\Phi$ on $\y$:
    \begin{align}
        | \partial^\mu ( a \circ \Phi ) | \leq | \D^{|\mu|} a|_{|\mu|,2}  ((|\Phi_j|)_{j \geq 1})^{\mu}\leq \Cat(|\mu|)((|\Phi_j|)_{j \geq 1})^{\mu}.\label{eq:gbound}
    \end{align}

    We finish the first part of the proof by expanding equation~\eqref{eq:inbetweenbound} using the bounds in equations~\eqref{eq:invbound} --~\eqref{eq:gbound} to obtain
    \begin{align*}
        &|\partial^\mu{A}(\bm{0})|_{2,2}\nonumber\\
        &\leq \sum_{\nu^{(1)}+\nu^{(2)}+\nu^{(3)}+\nu^{(4)}=\mu} \binom{\mu}{ \nu^{(1)},\nu^{(2)},\nu^{(3)}, \nuiv}
        |\nu^{(1)}|!\left( (|\D \Phi_j |_{2,2})_{j \geq 1} \right)^{\nu^{(1)}}\nonumber\\
        & \qquad \qquad    |\nu^{(2)}|!\left( |\D \Phi_j |_{2,2})_{j \geq 1} \right)^{\nu^{(2)}}  |\nu^{(3)}|!\sqrt{d}^{|\nu^{(3)}|}\left( |\D \Phi_j |_{2,2})_{j \geq 1} \right)^{\nu^{(3)}}\nonumber\\
        &\qquad\qquad\qquad \qquad  \Cat(|\nuiv|)((|\Phi_j|)_{j \geq 1})^{\nuiv} \nonumber \\
        &\leq  \sum_{\nuiv \leq \mu} \left[ \binom{\mu }{ \nuiv} \sqrt{d}^{\abs{\mu - \nuiv}} \left(( |\D \Phi_j |_{2,2})_{j\geq 1} \right)^{\mu - \nuiv}\right. \\
        &\qquad\qquad   \sum_{\nu^{(1)}+\nu^{(2)}+\nu^{(3)}=\mu-\nuiv} \binom{\mu - \nuiv }{ \nu^{(1)},\nu^{(2)},\nu^{(3)}} |\nu^{(1)}|! |\nu^{(2)}|! |\nu^{(3)}|!  \nonumber\\
        &\qquad\qquad\qquad\qquad\Cat(|\nuiv|) ((|\Phi_j|)_{j \geq 1})^{\nuiv}\Bigg]  \nonumber \\
        &\leq  \Cat(|\mu|)  \sum_{ \nuiv \leq \mu} \binom{\mu}{\nuiv} \frac{(|\mu| - |\nuiv| + 2)!}{2} \nonumber\\
        & \qquad\qquad((|\D \Phi_j|_{2,2} )_{j \geq 1})^{\mu - \nuiv} \sqrt{d}^{\,|\mu - \nuiv|} ((| \Phi_j| )_{j \geq 1})^{ \nuiv},
    \end{align*}
    where, on the last line, we employed Lemma~\ref{lem:multinomialsum} in Appendix~\ref{ap:combilemma} with $N=3$.

    To find the derivative $\left|\partial^\mu F(\bm{0}) \right|$, and finish the proof, we expand using the general Leibniz rule, such that equation~\eqref{eq:detbound}, and the analogous for $f$ of~\eqref{eq:gbound}, lead to
    \begin{align*}
        \left|\partial^\mu F(\bm{0}) \right| &\leq \sum_{\nui + \nuii = \mu} \binom{\mu }{ \nui, \nuii} | \partial^\nui \det(\D\Phi(\bm{0})) |  | \partial^\nuii (f \circ \Phi(\bm{0})) |\\
        &\leq \Cft(|\mu|)   \sum_{\nu \leq\mu} \binom{\mu }{ \nu} |\nu|!\sqrt{d}^{|\nu|}(|(\D \Phi_j|_{2,2})_{j \geq 1})^\nu ((|\Phi_j|)_{j \geq 1})^{\mu-\nu},
    \end{align*}
    for every $\mu\in\mathcal{F}$.
\end{proof}

With these bounds at hand, we define $(b_j(\x))_{j \geq 1}$ in the following corollary:
\begin{corollary}
    \label{lm:Afbound}
    For $\y \in Y $, let ${A}(\y; \x)$, $F(\y; \x)$ be defined by equations~\eqref{eq:hatAdefpois} and~\eqref{eq:hatfdefpois} respectively, and let Assumptions~\ref{ass:analytic} and~\ref{ass:phijpois} hold.
    Then, for a.e. $\x \in {D}$ and for every $\mu \in \mathcal{F}$, we have the bounds
    \begin{align}
        |\partial^\mu A(\bm{0};\x)|_{2,2} \leq ((b_j(\x))_{j \geq 1})^\mu f_A(|\mu|),
    \end{align}
    and
    \begin{align}
        |\partial^\mu F(\bm{0};\x)| \leq ((b_j(\x))_{j \geq 1})^\mu f_F(|\mu|),
    \end{align}
    where
    \begin{align}
        f_A(n)\coloneqq\Cat(n) \frac{(n+3)!}{3!} \text{, }\qquad f_F(n)\coloneqq\Cft(n) n!,
    \end{align}
    and
    \begin{equation}
        b_j(\x)\coloneqq |\Phi_j (\x)| + \sqrt{d}|\D\Phi_j(\x) |_{2,2}\label{eq:bj_phi}.
    \end{equation}
\end{corollary}
\begin{proof}
    The result follows from the application of Lemma~\ref{lem:multinomialsum} with $N=2$ to Lemma~\ref{thm:Afbound}, after we bound both $\sqrt{d}|\D\Phi_j|_{2,2}$ and $|\Phi_j|$ by $b_j$, $j \geq 1$.
\end{proof}
To verify the assumptions of Theorem~\ref{thm:lpsummability}, we are left with the task to verify conditions~\eqref{eq:rhoArhoFass} --~\eqref{eq:assKbKc} and to find a positive sequence $(\rho_j)_{j\geq 1}$ with $\rho_j > 1$ satisfying condition~\eqref{eq:Kbassumption}.
To attain this goal, we need to bound $b_j(\x)$, $j \geq 1$, in order to determine $\Kt < 1$ in~\eqref{eq:Kbassumption} and finally show
\begin{align}
    g_A(\Kb) &= \sum_{n  \geq 1} \Cat(n) \binom{n+3}{3} \Kb^n\\
    &<A_{\min}
    = a_{\min}\sigma_{\min}^4, \label{eq:gAnonexplicit}
\end{align}
where the value of $A_{\min}$ follows from Lemma~\ref{lem:courantfish2pois} combined with the definition of $A(\y,\cdot)$ from equation~\eqref{eq:hatAdefpois}.
Therefore, we have
\begin{align}
    g_A(\Kb) <  a_{\min}\sigma_{\min}^4  = A_{\min}\label{eq:gAdefshape},
\end{align}
and
\begin{equation*}
    \min\{\rho_A\rho_F, \rho_A\} > \Kt.
\end{equation*}
Here $\sigma_{\min}$ is the lower bound on the singular values of $\D\Phi^{-1}$ and follows from Lemma~\ref{lem:courantfish2pois}.

To bound $b_j(\x)$, $j \geq 1$, we are dependent on bounding $\Phi_j(\x)$ and $\D\Phi_j(\x)$.
We will formulate these bounds in the next section.

\begin{remark}
    In the special case where the sequences $\Ca(n)$ and $\Cf(n)$ are bounded, and therefore $\Cat(n)\equiv \Cat$ and $\Cft(n)\equiv\Cft$ are constant, we can expand equation~\eqref{eq:gAnonexplicit} explicitly into
    \begin{align*}
        g_A(\Kb) = \Cat \left( \frac{1}{(1-\Kt)^4} -1\right).
    \end{align*}
    Moreover, equation~\eqref{eq:assKbKc} is satisfied by Remark~\ref{rem:Kbinterval}.
\end{remark}
\begin{remark}
    We observe that sharper constants in the convergence estimate could be obtained by explicitly adapting the proof of Theorem~\ref{thm:l2summability} to the specific case of parametric domains considered in this section.
    This would improve the constants whilst leading to the same summability exponent for the Taylor coefficients.
    However, the summability exponent for the Taylor coefficients would be the same as the one we obtained from the more general result of Theorem~\ref{thm:l2summability}.
\end{remark}
\begin{remark}
    From equation~\eqref{eq:gAdefshape}, we can deduce that, as described in Remark~\ref{rem:seperation1}, the important quantity here is the amount of relative variation to $g_A(\Kt)/(a_{\min}\sigma_{\min}^4)$, where we see how different domain mappings affect the constants.
\end{remark}

\subsection{Bounds for the domain mappings}\label{subsec:transformation-methods}
With the appropriate bounds on $A$ and $F$ at hand, we need to bound $b_j(\x)$ from equation~\eqref{eq:bj_phi} to find an appropriate sequence $\rho_j$ that satisfies equation~\eqref{eq:Kbassumption}.
In this section, we will compute these bounds for both the mollifier mapping and harmonic extension as introduced in Chapter~\ref{ch:uncertainty-quantification-for-paramterized-domains}.
Moreover, we will formulate appropriate sequences $\rho_j$ for both mapping approaches.

The mollifier approach will offer several advantages when studying the theoretical properties of this mapping, and we will explore this for Theorem~\ref{thm:l2summability} in Section~\ref{subsubsec:explicittransformation}.
The smoothing properties of the harmonic extension bring numerical advantages compared to the mollifier approach, as we will see in the numerical experiments.
On the other hand, rigorously deriving the bounds required to apply Theorem~\ref{thm:l2summability} is more challenging due to the lack of a general explicit expression of $\Phi(\y,\x)$ in terms of $\x$.
\begin{remark}
    To find a sequence $\rho_j$ that satisfies the assumptions of Theorem~\ref{thm:l2summability}, we need, by Remark~\ref{rem:Kbinterval}, that
    \begin{equation}
        \brhosum \leq \min\left\{ \rho_A \rho_F, \rho_A \right\}.\label{eq:nessrhosumrhoarhof2}
    \end{equation}
    In our case, this gives rise to an implicit restriction on the regularity of the coefficients $a$ and $f$ in equation~\eqref{eq:shapepoisson}, which we have to check after having chosen $\left\{ \rho_j \right\}_{j\geq 1}$.
\end{remark}

\subsubsection{Mollifier transformation}
\label{subsubsec:explicittransformation}
In this section, we examine a transformation that maps the nominal domain ${D}$ to the parameterized domain $\Dy$ by employing the mollifier transformation outlined in Section~\ref{subsec:mollifier-mapping}.
To verify the assumptions of Theorem~\ref{thm:lpsummability}, we need to calculate an explicit bound on the Jacobian matrices $\D\Phi_j(\x)$, $j \geq 1$, which is given by the following lemma:
\begin{lemma}\label{lem:Dphibound}
For the transformation given by equation~\eqref{eq:mollifiertransf} with the parameter-dependent radius as in equation~\eqref{eq:rdef}, such that $\Phi_j(\x)$ with $j\geq 1$ is given by equation~\eqref{eq:frameworkmapping}, we have the bound
\begin{equation}
    |\D\Phi_j(\x)|_{2,2} \leq \chibarone | \psi_j(\theta(\x)) | +  \chibartwo |\psi_j'(\theta(\x))  |, \qquad \text{for a.e. } \x \in {D}, \label{eq:dphijbound}
\end{equation}
with
\begin{align}
    \chibarone &= \sup_{\x \in \mathrm{supp}(\chi)} \frac{1}{\sqrt{2}} \Bigg( |\nabla_{\x} \chi(\x)|^2  +\frac{\chi(\x)^2}{|\x|^2}  +\nonumber \\
    &\qquad \qquad \qquad \sqrt{\left( |\nabla_{\x} \chi|^2  +\frac{\chi(\x)^2}{|\x|^2}   \right)^2-\frac{4\chi(\x)^2}{|\x|^4}  \left( \nabla_{\x}\chi(\x)\cdot \x \right) ^2}  \Bigg)^{\frac{1}{2}},
\end{align}
and
\begin{align}
    \chibartwo &= \sup_{\x \in \mathrm{supp}(\chi)}  \frac{\chi\left(\x\right)}{|\x| }.
\end{align}
\end{lemma}
\begin{proof}
    The proof relies on establishing a bound of the form
    \begin{align*}
        |\D\Phi_j(\x)|_{2,2} \leq| \psi_j(\theta(\x)) ||  A_1(\x) |_{2,2}  +  |  \psi_j'(\theta(\x))  || A_2(\x)|_{2,2},
    \end{align*}
    for a.e. $\x \in D$, and calculating the largest singular values of $A_1(\x)$ and $A_2(\x)$, see Appendix~\ref{ap:Dphibound_boundpf} for details.
\end{proof}

For the interior affine mollifier from Definition~\ref{def:interior_mollifier}, we can compute the constants $\chibarone$ and $\chibartwo$ explicitly:
\begin{corollary}[Bounds for interior mollifier]
    \label{cor:chi_bounds_interior_corollary}
    For the interior affine mollifier~\eqref{eq:interior_mollifier}, the constants in Lemma~\ref{lem:Dphibound} are given by
    \begin{equation}
        \chibarone  = \frac{4}{3{r_0^-}} \sqrt{ \left( 1+ \frac{L_{r_0} ^2}{{r_0^-}^2}    \right) + \left(1 - \frac{r_0^-}{4r_0^-}\right)^2 },\label{eq:chibarone}
    \end{equation}
    and
    \begin{equation}
        \chibartwo = \frac{4}{{3r_0^-}} \left(1 - \frac{r_0^-}{4r_0^-}\right),\label{eq:chibartwo}
    \end{equation}
    where $L_{r_0}$ is the Lipschitz constant of $r_0$.
\end{corollary}
The proof follows by direct computation, which we present in Appendix~\ref{ap:Dphibound_boundpf}.

Now, we expand on the special case where the functions $\psi_j$ in equation~\eqref{eq:rdef} are wavelets as introduced in Chapter~\ref{ch:uncertainty-quantification-for-paramterized-domains}.
To find the optimal convergence rate of the Taylor coefficients~\eqref{eq:taylorseq} of the solution to~\eqref{eq:transformedvarformpois}, we propose a sequence $\rho_j$ to verify the assumptions of Theorem~\ref{thm:lpsummability}.
Lemma~\ref{lem:poitwisesum}, together with Assumption~\ref{ass:mollifier}, ensure that the domain mapping fulfills Assumption~\ref{ass:phijpois}.

To this aim, we turn our attention to the variational formulation on the reference domain, equation~\eqref{eq:transformedvarformpois}.
We remind the reader that Assumption~\ref{ass:phijpois} when considered with Assumption~\ref{ass:analytic}, ensure the fulfillment of Assumption~\ref{ass:afanalytic}.
Together with the bounds from Lemma~\ref{lem:poitwisesum},  this guarantees that, for every $\y\in Y$, $A(\y; \x)$ and $F(\y;\x)$ as defined in~\eqref{eq:hatAdefpois}-\eqref{eq:hatfdefpois} are in $W^{1,\infty}(D)$ and $L^2(D)$, respectively, with $\y$-independent norm bounds.

To verify the assumptions of Theorem~\ref{thm:lpsummability}, we recall that, on the basis of Corollary~\ref{lm:Afbound}, the sequence $(b_j(\x))_{j\geq 1}$ appearing in~\eqref{eq:AFder} is given by~\eqref{eq:bj_phi}, provided we use $\lambda$ as index.
Then, for the sequence $(\rho_{\lambda})_{\lambda}$ we propose, for any $\beta < \alpha$,
\begin{equation}
    \rho_\lambda \coloneqq 1 + \frac{\mathcal{B}_M(1-2^{\beta - \alpha})2^{(\beta - 1)|\lambda|}}{1-2^{-\alpha}},   \label{eq:rhoprop}
\end{equation}
where\footnote{We note that the definition of $\mathcal{B}_{M}$ given here is slightly different from the version published in~\cite{vanharten2024}. This is due to a small typographical error in the published version, and the following derivations here have been changed accordingly.}
\begin{align}
    \mathcal{B}_M = \frac{g_A^{-1}(\amin\sigma_{\min}^4)}{1 + \sqrt{2}\chibarone + \sqrt{2}\frac{M_d}{M}\chibartwo\frac{1-2^{-\alpha}}{1-2^{-(\alpha-1)}}} \frac{1}{r_0^- \vartheta}  - 1. \label{eq:B}
\end{align}
We need to ensure that $\mathcal{B}_M>0$ by choosing a small enough value for the maximal shape variation $\vartheta$.
In~\eqref{eq:B}, we remind the reader that $\sigma_{\min}$ is the lower bound on the singular values of the Jacobian matrix.

To check that~\eqref{eq:Kbassumption} is fulfilled, we expand the sum in~\eqref{eq:phidefpois} and treat the resulting terms separately:
\begin{align*}
    &\sum_\lambda \rho_\lambda|\D\Phi_\lambda(\x)|_{2,2}\\
    &\qquad= \sum_\lambda \left( 1 + \frac{B(1-2^{\beta - \alpha})2^{(\beta - 1)|\lambda|}}{(1-2^{-\alpha})\left(\bar{\chi}_1  + \bar{\chi}_2\frac{M_d}{M}\right)} \right)|\D\Phi_\lambda|_{2,2}\\
    &\qquad= \sum_\lambda |\D\Phi_\lambda(\x)|_{2,2} + \sum_\lambda \frac{\mathcal{B}_M(1-2^{\beta - \alpha})2^{(\beta - 1)|\lambda|}}{(1-2^{-\alpha})} |\D\Phi_\lambda(\x)|_{2,2}.
\end{align*}
First, making use of Lemmas~\ref{lem:poitwisesum} and~\ref{lem:Dphibound}, we bound the first term
\begin{align*}
    \sum_\lambda  |\D\Phi_\lambda(\x)|_{2,2}&\leq  \sum_\lambda \left(\bar{\chi}_1| \psi_j(\theta(\x)) | + \bar{\chi}_2|  \psi_j'(\theta(\x))  |\right)\nonumber\\
    &\leq  r_0^-\vartheta \left( \bar{\chi}_1 + \bar{\chi}_2\frac{M_d}{M} \frac{1-2^{-\alpha}}{1-2^{-(\alpha-1)}}    \right),
\end{align*}
and the second term
\begin{revenv}
    \begin{align*}
        &\sum_\lambda   \frac{\mathcal{B}_M(1-2^{\beta - \alpha})2^{(\beta - 1)|\lambda|}}{   (1-2^{-\alpha})} |\D\Phi_\lambda(\x)|_{2,2}\\
        &\qquad\leq\sum_{l \geq 0} r_0^-\mathcal{B}_M(1-2^{\beta - \alpha})2^{(\beta - 1)l} \left(\bar{\chi}_1 2^{-l} +\bar{\chi}_2 \frac{M_d}{M}\right) 2^{-(\alpha-1) l}\\
        &\qquad\leq r_0^- \vartheta  \mathcal{B}_M \left(\bar{\chi}_1 + \bar{\chi}_2\frac{M_d}{M}\right).
    \end{align*}
\end{revenv}
Combining both, we get
\begin{align*}
    &\sum_\lambda \rho_\lambda|\D\Phi_\lambda(\x)|_{2,2} \nonumber\\
    &\qquad\leq r_0^-\vartheta \left( \bar{\chi}_1 + \bar{\chi}_2\frac{M_d}{M} \frac{1-2^{-\alpha}}{1-2^{-(\alpha-1)}}  + \mathcal{B}_M\left(\bar{\chi}_1 + \bar{\chi}_2\frac{M_d}{M}\right)  \right),
\end{align*}
see also~\eqref{eq:pointwiseone} and~\eqref{eq:pointwisetwo}.
In a similar manner, we bound $\sum_\lambda \rho_\lambda|\Phi_\lambda|$ so that, altogether, we obtain
\begin{align}
    \sum_{\lambda} \rho_\lambda b_\lambda(\x) &\leq r_0^-\vartheta \Bigg( 1+ \sqrt{2} \bar{\chi}_1 +  \sqrt{2}\bar{\chi}_2\frac{M_d}{M} \frac{1-2^{-\alpha}}{1-2^{-(\alpha-1)}} + \nonumber\\
    &\qquad\qquad\qquad\qquad\qquad\mathcal{B}_M \Big(1+ \sqrt{2}\chibarone + \sqrt{2}\chibartwo \frac{M_d}{M}\Big)  \Bigg)\nonumber\\
    &\coloneqq \Kb, \label{eq:endmollifiercalculation}
\end{align}
for $\vartheta$ small enough to satisfy~\eqref{eq:rhoArhoFass}.

Because of our choice of $\mathcal{B}_M$ as from~\eqref{eq:B}, we have $\Kb < g_A^{-1}(\amin\sigma_{\min}^4)$, so that equation~\eqref{eq:gAdefshape}, and therefore~\eqref{eq:assKbKc}, are satisfied.
We also have
\begin{equation*}
    \rho_\lambda \sim 2^{(\beta-1)|\lambda|},
\end{equation*}
which can be reordered to obtain, for all $j \geq 1$,
\begin{equation*}
    \rho_j \sim j^{(\beta-1)},
\end{equation*}
using a single index~\cite{bachmayr2017a}.

This sequence satisfies $(\rho_j^{-1})_{j\geq 1} \, \in  \, \ell^q(\mathbb{N})$ with $q>(\beta-1)^{-1}$, so that we can use Theorem~\ref{thm:lpsummability} to obtain $(\anorm{t_\mu})_{\mu \in \mathcal{F}} \, \in \, \ell^p(\mathcal{F})$ for $p>\frac{2(\beta-1)^{-1}}{2+(\beta-1)^{-1}}=\frac{2}{2(\beta-1) + 1}$.
This implies that $(\anorm{t_\mu})_{\mu \in \mathcal{F}}$ converges with rate $s=\frac{1}{p} = \frac{2(\beta-1) + 1}{2} = (\beta-1) + \frac{1}{2} = \beta - \frac{1}{2}$, for any $\beta < \alpha$ and $\alpha > 2$.
Therefore, we have the following corollary:

\begin{corollary}\label{cor:mollifiersummability}
    For the solution to the parameterized PDE~\eqref{eq:transformedvarformpois} with coefficients satisfying Assumption~\ref{ass:analytic}, we consider a mollifier mapping~\eqref{eq:mollifiertransf} with a mollifier that fulfills Assumption~\ref{ass:mollifier}.
    We use wavelets, as defined in~\eqref{eq:waveletdef}, with $\alpha > 2$ in the radius expansion~\eqref{eq:rdef}, along with a maximal shape variation $\vartheta$, $\mathcal{B}_M > 0$ and $\Kt < \max\{\rho_A \rho_F, 1\}$.
    Under these conditions, we conclude that $(\anorm{t_\mu})_{\mu \in \mathcal{F}} \in \ell^p(\mathcal{F})$, with $\frac{1}{p} < \alpha - \frac{1}{2}$.
\end{corollary}

\begin{remark}[General support basis functions]
    With a technique similar to Remark 2.4 in~\cite{bachmayr2017a}, one can verify that, for general support basis functions, we cannot improve over previous results~\cite{chkifa2015,hiptmair2018}, where a summability exponent of $s < \alpha - 1$ for the Taylor coefficients was shown.
\end{remark}


\subsubsection{Harmonic transformation}
\label{subsubsec:harmonic}
To show that we can use the harmonic mapping and satisfy the assumptions of Theorem~\ref{thm:lpsummability}, we will focus our attention on the circular domain $D=\{ \x \in \mathbb{R}^2 : |\x| \leq r_0\}$, with $r_0 > 0$ in the remainder of this section.
Similar to the mollifier transformation, and in view of~\eqref{eq:bj_phi}, we look for pointwise bounds on the Jacobian matrices.

To bound the norm of the Jacobian matrix of each partial transformation $\Phi_j$, we proceed in two steps: (I) we bound the supremum of the Jacobian matrix by considering its value at the boundary only, applying the maximum principle, and (II) we exploit the Dirichlet-to-Neumann map to estimate the derivatives at the boundary in the tangential and normal directions.

\begin{itemize}
    \item[(I)]
    To bound the norm of the Jacobian matrix of the partial transformation, we bound it, for a.e. $\x \in D$, by the pointwise 1-norm of the elementwise transformations:
    \begin{align*}
        |\D\Phi_j(\x)|_{2,2}
        \leq\sum_{\kappa \in \{x, y\}}  |\nabla \Phi_j^\kappa(\x)|_1.
    \end{align*}
    As $\Phi_j$ is a weakly harmonic function, the derivatives of $\Phi_j$ are weakly harmonic as well, and the absolute value of the derivatives of $\Phi_j$ and their sum are weakly subharmonic.
    Hence, we can apply the maximum principle, see, for example~\cite{serrin2011}, to conclude that:
    \begin{align}
        \sup_{\x \in D} \sum_{\kappa \in \{x, y\}}  |\nabla \Phi_j^\kappa (\x)|_1 \leq \sup_{\x \in \partial D} \sum_{\kappa \in \{x, y\}}  |\nabla \Phi_j^\kappa(\x)|_1. \label{eq:partialmaxprinc}
    \end{align}
    \item[(II)]
    To evaluate the right-hand side of~\eqref{eq:partialmaxprinc} at the boundary, we estimate, for a.e. $\x \in D$,
    \begin{align*}
        |\nabla \Phi_j^\kappa(\x)|_1 \leq  \sqrt{2} |\nabla \Phi_j^\kappa(\x)|,
    \end{align*}
    where we remind that $|\cdot|$ is the Euclidean norm.
    Now, we calculate the derivatives explicitly at the boundary in both the radial and angular directions, that is, the normal and tangential derivatives at the boundary, respectively.
    For the angular derivative, we can differentiate the boundary condition in~\eqref{eq:harmonicext} to obtain $ \frac{\partial \phi_j^\kappa}{\partial \theta} = (f_\kappa\psi_j)'(\theta)$, for $\theta \in [0,2\pi)$ and $\kappa\in\{x,y\}$.

    To complete the computation of the gradient, we observe that the radial derivative of $\Phi_j^\kappa$ equals to the Dirichlet-to-Neumann (DtN) map of $\phi_j^\kappa$, $j\geq 1$.

    For circular domains, the DtN map of the Laplacian is given by $\sqrt{-\Delta_{\partial D}}$, the square root of the Laplace-Beltrami operator on the boundary circle~\cite{girouard2022}.
    Because our domain is circular, we pick up a factor of $\frac{1}{r_0}$ due to the curvature of the circle, and we can expand using the symbol of the operator and the Fourier transform $\mathcal{F}(\cdot)$,
    \begin{align*}
        \sqrt{-\Delta_{\partial D}} \phi_j^\kappa
        &=\frac{1}{r_0}\int_\mathbb{R} |\xi| \mathcal{F}(\phi_j^\kappa )(\xi) e^{-i \xi x} \d\xi\\
        &=\frac{1}{r_0}\int_\mathbb{R} i\xi (-i) \sign(\xi) \mathcal{F}(\phi_j^\kappa )(\xi) e^{-i \xi x} \d\xi\\
        &=\frac{1}{r_0}\int_\mathbb{R} \mathcal{F}\left(\frac{\d}{\d x}\mathcal{H}(\phi_j^\kappa )\right) (\xi) e^{-i \xi x} \d\xi\\
        &=\frac{1}{r_0}\frac{\d}{\d x}\mathcal{H}(\phi_j^\kappa ),
    \end{align*}
    where $\mathcal{H}(\cdot)$ denotes the Hilbert transform~\cite{hahn1996}.

    Now, we remember that, for any signal $f$, the signal $f^A\coloneqq f+i\mathcal{H}(f)$ is analytic, where analyticity is meant in the signal processing sense; it has no negative frequency components.
    Moreover, we recall that the instantaneous amplitude $A(f^A)(x)$ of any analytic signal $f^A$ is given by the modulus of the signal itself.
    Therefore, we observe that, for every $\theta\in[0,2\pi)$ and $\kappa \in \{x,y\}$,
    \begin{align}
        \left| \nabla \phi_j^\kappa(\theta)  \right|&\leq \sqrt{2} \max\{1, r_0^{-1}\} \left|(f_\kappa\psi_j)'(\theta) + i \mathcal{H}\left( (f_\kappa\psi_j)' \right)(\theta)  \right| \nonumber\\
        &= \sqrt{2} \max\{1, r_0^{-1}\} A((f_\kappa\psi_j)')(\theta) \nonumber\\
        &\leq \sqrt{2} \max\{1, r_0^{-1}\} \left( A(\psi_j)(\theta) + A(\psi_j')(\theta) \right),
    \end{align}
    where we have used the definition of $f_\kappa$, which has an amplitude of 1.
    From this, for a.e. $\theta \in [0,2\pi)$, we obtain the bound
    \begin{align*}
        |\D\Phi_j(\theta)|_{2,2} \leq  4 \max\{1, r_0^{-1}\}  \left( A(\psi_j)(\theta) + A(\psi_j')(\theta) \right).
    \end{align*}
    with $\x$ on the boundary $\partial D$, corresponding to the angle $\theta$.

\end{itemize}
The sum $\sum_{j \geq 1} \rho_j|\D\Phi_j(\x)|_{2,2} $ is bounded from above by the weakly subharmonic function $\sum_{j\geq 1}  \sum_{\kappa \in \{x, y\}}  |\nabla \phi_j^\kappa(\x)|_1$ and its maximum is attained at the boundary as well.
Therefore, we can repeat \rev{steps} (I) and (II) to obtain the bound
\begin{align}
    &\hspace{-20pt}\sup_{\x \in D}\sum_{j\geq 1} \rho_j |\D\Phi_j(\x)|_{2,2}  \nonumber\\
    &\hspace{-20pt}\,\,\leq \sup_{\theta \in [0,2\pi)}  4 \max\{1, r_0^{-1}\} \left(  \sum_{j \geq 1}\rho_j A(\psi_j)(\theta) + \sum_{j \geq 1}\rho_j A(\psi_j')(\theta) \right). \label{eq:rhodphisum}
\end{align}
Similarly, since $\Phi_j$ is harmonic, we can repeat the same argument and obtain
\begin{align}
    \sup_{\x \in D}\sum_{j\geq 1} \rho_j |\Phi_j(\x)|\leq \sup_{\theta \in [0,2\pi)} \sum_j \rho_j |\psi_j(\theta)|.\label{eq:rhophisum}
\end{align}
Finally, combining equations~\eqref{eq:rhodphisum} and~\eqref{eq:rhophisum}, we arrive at the following lemma:
\begin{lemma}\label{lem:harmboundbj}
For the harmonic transformation given by equation \eqref{eq:harmonicext}, with $b_j(\x)$ from equation~\eqref{eq:bj_phi} and a positive sequence $(\rho_j)_{j \geq 1}$ on a circular, two-dimensional reference domain, we have the bound
\begin{align*}
    &\sup_{\x \in D}\sum_j \rho_j b_j(\x) \\ &\qquad\leq \sup_{\theta \in [0,2\pi)} \Bigg( \left(4\sqrt{2} \max\{1, r_0^{-1}\} + 1\right) \sum_{j}\rho_j A(\psi_j)(\theta) + \\  &\qquad\qquad\qquad\qquad\qquad\qquad\qquad4\sqrt{2} \max\{1, r_0^{-1}\} \sum_{j}\rho_j A(\psi_j')(\theta) \Bigg).
\end{align*}
\end{lemma}
Similarly to Subsection~\ref{subsubsec:explicittransformation}, we focus on the wavelet expansion~\eqref{eq:waveletdef}, with decay parameter $\alpha>2$, Lipschitz mother wavelet, and we index the wavelets using $\lambda$ instead of $j$.
To show the desired estimate on $\sum_{\lambda} \rho_\lambda b_\lambda (\x)$, we set, similarly to equations~\eqref{eq:Mbound} and~\eqref{eq:Mbound3}:
\begin{equation}
    M_H\coloneqq\sup_{\theta \in [0,1)} \sum_{k \in \mathbb{Z}}A(\Psi)(\theta - k), \label{eq:Mbound2}
\end{equation}
and
\begin{equation}
    M_{d,H}\coloneqq\sup_{\theta \in [0,1)} \sum_{k \in \mathbb{Z}}A(\Psi')(\theta - k), \label{eq:Mbound4}
\end{equation}
and we proceed as in Subsection~\ref{subsubsec:explicittransformation}, replacing $M$ and $M_d$ with $M_H$, $M_{d,H}$ whenever applicable.
We have to be careful, though, as the Hilbert transform does not preserve the locality properties of $\Psi$.
However, if the mother wavelet $\Psi(\theta)$ has $n$ vanishing moments, the derivative of decays with $\left|  \Psi' (\theta) \right| \leq \mathcal{O}(n^{-1-n})$~\cite{chaudhury2011}, which gives us the required summability when we periodize to obtain a wavelet system on the circle.
Hence, we recover Lemma~\ref{lem:poitwisesum} with the aforementioned replacements.

Now, we take $\rho_\lambda$ as defined in equation~\eqref{eq:rhoprop}, where $\mathcal{B}_M$ is replaced by
\begin{equation}
    \mathcal{B}_H = \frac { \splitfrac{ \frac{1}{\vartheta}g_A^{-1}(\amin\sigma_{\min}^4) - 4\sqrt{2}\max\{1, r_0 \}   -}{\qquad\qquad\qquad\left(4\sqrt{2}\max\{1, r_0 \} + r_0 \right)   \frac{M_{d,H}}{M_H}\frac{1-2^{-\alpha}}{1-2^{-(\alpha - 1)}}}   }{4\sqrt{2}\max\{1, r_0 \}\left( 1+\frac{M_{d,H}}{M_H} \right) + r_0  }.\label{eq:Bhdef}
\end{equation}
Similarly to $\mathcal{B}_M$, we should be careful in choosing $\vartheta$ small enough such that we keep $\mathcal{B}_H$ positive.
By repeating the calculations from equation~\eqref{eq:rhoprop} to Corollary~\ref{cor:mollifiersummability}, we arrive at the following statement:
\begin{corollary}\label{cor:harmonicsummability}
For the solution to the parameterized PDE~\eqref{eq:transformedvarformpois} satisfying Assumption~\ref{ass:analytic}, and a harmonic mapping~\eqref{eq:harmonicext}, and using wavelets as in~\eqref{eq:waveletdef} for the radius expansion~\eqref{eq:rdef} with maximal shape variation $\vartheta$ such that $\mathcal{B}_H>0$ and $\Kt < \max\{\rho_A\rho_F,1\}$, we have that
$(\anorm{t_\mu})_{\mu \in \mathcal{F}}$ $\in \ell^p(\mathcal{F})$, with rate $\frac{1}{p} < \alpha - \frac{1}{2}$,  $\alpha > 2$.
\end{corollary}

\begin{remark} \label{rem:sepofvar}
The results leading up to Lemma~\ref{lem:harmboundbj} that have been obtained using the symbol of $\sqrt{-\Delta_{\partial D}}$ can be obtained via direct computations of the exact solution of equation~\eqref{eq:harmonicext} which, in turn, can be obtained via separation of variables, see for instance~\cite{natalini2008}.
\end{remark}

\begin{remark}
    For \rev{non-circular domains} with Lipschitz boundary, we cannot rely on an explicit formula for the DtN map similarly to the approach outlined in this section; see, for instance,~\cite{girouard2022}.

    Direct computations, as outlined in Remark~\ref{rem:sepofvar}, do not seem to help draw useful conclusions either.
    We believe that the general result should follow from the quasi-local properties of the DtN map.
    Still, these are difficult to find in the literature, which focuses more on global properties expressed in terms of norms.

    \begin{revenv}
        However, for practical purposes, we note that the applicability of the harmonic mapping goes even beyond the star-shaped setting presented here if one chooses an appropriate series expansion for the boundary, similar to equation~\eqref{eq:rdef}.
    \end{revenv}
\end{remark}