In this chapter, we demonstrate how the findings from Chapter~\ref{ch:exploiting-locality-in-polynomial-approximations} can be utilized to address elliptic problems on parametric domains.
We highlight the advantages of using functions with localized supports to represent parametric boundaries and, more broadly, domain variations.
Specifically, we discuss how this approach can improve convergence rates for polynomial surrogate models.
Furthermore, we present numerical results that, in addition to illustrating the theory, provide further insight into computational approaches for handling parameter-dependent geometry.

First, we start with a literature review of the numerical treatment of parameterized domains in Section~\ref{sec:state-of-the-art2}.
In Section~\ref{sec:modelproblem}, we introduce the stationary diffusion equation on parameterized domains, and we use the results from the previous chapter to investigate the summability properties of the Taylor coefficients for the solution on the nominal configuration.
There, we also discuss different mapping approaches to map to the nominal configuration.
In Section~\ref{sec:numerical_taylor}, we discuss the implementation of our numerical experiments and numerically illustrate the theoretical results for the parameterized domain problem.
We end with some final remarks and conclusions in Section~\ref{sec:discussion_taylor_num}.