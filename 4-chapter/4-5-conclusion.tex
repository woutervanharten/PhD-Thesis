In this chapter, we have leveraged the general theory developed in Chapter~\ref{ch:exploiting-locality-in-polynomial-approximations} to derive convergence bounds for Taylor approximations for the parametric domain setting introduced in Chapter~\ref{ch:uncertainty-quantification-for-paramterized-domains}.
In Section~\ref{sec:modelproblem}, we have discussed two mapping approaches: using an affine mollifier and a mapping defined through a harmonic extension problem.
Then, presented the numerical results supporting the theoretical findings, together with their interpretation in Section~\ref{sec:numerical_taylor}.

Analytically, the mollifier approach has an advantage over the harmonic mapping because we can establish pointwise bounds on the Jacobian matrix $\D\Phi$ by explicitly calculating it for star-shaped domains.
This is in contrast to the harmonic extension, where we have shown similar bounds only for circular domains.
However, intuitively, similar results are expected for other classes of star-shaped domains.

In contrast to the analytical preference for the mollifier mapping, the harmonic mapping has significant numerical advantages over the first one.
First, the harmonic mapping is such that the spectral norm of the Jacobian matrix $\D\Phi$ is relatively small compared to one for the mollifier mapping.
Moreover, wavelets deep into the wavelet structure are such that the support of $\D\Phi_j$ is strongly localized near the domain boundary.
Therefore, calculating the Taylor coefficients requires a fine mesh near the boundary of the computational domain to resolve the fine details in the small wavelets.
The mesh can be kept rough on its interior to reduce the computational burden.

Instead of calculating the transformation by solving a partial differential equation for each dimension separately, other PDE-based approaches to build the mapping act on all coordinates simultaneously.
One example of such a mapping is defined through the linear elasticity equations~\cite{cizmas2008,dwight2009}, which can be interpreted as elastically stretching the nominal onto the parameterized domain.
Elasticity equations offer even more control over the smoothing properties of the transformation by using piecewise constant material properties.
Although this thesis focuses on star-shaped domains, a significant advantage of PDE-based mapping constructions is their geometrical flexibility, including possible extensions to non-star-shaped domains.

The theoretical convergence rate obtained in Theorem~\ref{thm:lpsummability} is supported by numerical results for the parametric domain problem~\eqref{eq:variational_formulation_y_poission}.
The numerical experiments show that both the choice of basis expansion and the mapping approach significantly impact the observed convergence rate.
Moreover, the critical role of the mesh in the process showcases the possible benefits of parameter-adapted meshes, where the mesh is refined adaptively, depending on the value of the parameter.
