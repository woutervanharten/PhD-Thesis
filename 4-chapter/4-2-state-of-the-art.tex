In the application to parametric domains, we adopt the mapping approach outlined in Chapter~\ref{ch:uncertainty-quantification-for-paramterized-domains}, leading to a problem that fits within the framework~\eqref{eq:pde}.
This involves transforming the PDE on the parameterized domain to a parameterized PDE on a nominal (reference) domain.
The smoothness of the solution on the reference configuration with respect to the parameter has been analyzed in~\cite{castrillon-candas2016,harbrecht2016,multerer2019} for the linear elliptic equation.
Smoothness has also been analyzed for the Helmholtz transmission problem~\cite{hiptmair2018, hiptmair2024}, the linear parabolic equation~\cite{castrillon-candas2021}, the stationary Navier-Stokes equations~\cite{cohen2018}, and Maxwell's equations in frequency domain~\cite{jerez-hanckes2017}.

While all these works prove smoothness in terms of the global properties of the parametric boundary, in the spirit of the general results in~\cite{zhou2014, chkifa2015}, the present work relies on \textsl{pointwise} properties, in the spirit of~\cite{bachmayr2017,bachmayr2017a,dung2022,zech2018}.
This shows that parametrizing the boundary, or more generally, the domain mapping, using functions with localized supports leads to higher convergence rates for surrogates based on truncated Taylor expansions.
We highlight that, in the case of parametric interface problems, one might also be interested in the solution for the physical, rather than the nominal, configuration.
In that case, the dependence of the solution on the parameter is not smooth~\cite{motamed2013,scarabosio2017,scarabosio2022}.

When discussing the smoothness of the solution on the nominal configuration, we analyze theoretically and compare numerically two approaches to construct a domain mapping starting from the boundary parametrization, namely an explicit expression using a mollifier and the harmonic extension~\cite{li2001,xiu2006}, as we introduced in Chapter~\ref{ch:uncertainty-quantification-for-paramterized-domains}.
The latter shows better localization properties in the numerical results and is better suited for complex geometries.
We analyze it as a prototype for other PDE-based methods, such as those based on elasticity equations; see, for instance,~\cite{cizmas2008,dwight2009}.
Another possibility, acting on the discretized PDE, is to use isogeometric analysis (IGA), possibly with boundary elements~\cite{dolz2022,dolz2023}.