The first results on convergence rates for polynomial surrogates of the parameter-to-solution map considered the case of a finite-dimensional parameter; see, for instance,~\cite{zhou2014,babuska2004,xiu2005}.
Convergence results for a parameter with a countable number of entries, in other words, dimension-independent rates, were first obtained in~\cite{todor2007} and improved in~\cite{cohen2010,cohen2011}.
These works study the model problem~\eqref{eq:pde} when the right-hand side is independent of the parameter and the coefficient depends on it in an affine way, that is $A(\y)=a(\y)I_d$, where $I_d$ is the identity matrix and $a(\y)=a_0 + \sum_{j\geq 1}y_j\psi_j$, for given $a_0, \left\{\psi_j\right\}_{j\geq 1}\in L^{\infty}(D)$ such that $a$ is uniformly positive and bounded (extension to matrix-valued coefficients is straightforward).
The aforementioned works obtain convergence rates on polynomial chaos and collocation-based surrogates from estimates of Taylor coefficients (with respect to the parameter).

A significant step forward was taken in~\cite{chkifa2015}, where convergence rates for Legendre instead of Taylor coefficients were proved to hold for a much larger class of problems, namely problem~\eqref{eq:pde} with smooth but potentially highly nonlinear dependence of $A(\y)$ and $F(\y)$ on the parameter, and a broader class of PDEs. We refer to~\cite{cohen2015} and~\cite[Ch. 2-4]{adcock2022} for a review of the so-far mentioned results.

All these works are based on the fact that the solution to the PDE, as a function of the parameter, admits a smooth extension on sets of the form $\mathcal{O}=\otimes_{j\geq 1}\mathcal{O}_j$.
These sets satisfy the condition that $[-1,1]\subset\mathcal{O}_j$ and that $\mathcal{O}_j$ is an open set on the real line or in the complex plane.
The sizes of the sets $\mathcal{O}_j$ are determined by the sensitivity of the parametric input with respect to $y_j$.
This sensitivity is measured using Banach space norms.

More concretely, consider the case where the coefficient and right-hand side in~\eqref{eq:pde} depend on a parametric input $\mathfrak{p}(\y) = \mathfrak{p}_0 + \sum_{j\geq 1}y_j\psi_j$, taking values in a Banach space $X$.
In the affine case, we have $\mathfrak{p}(\y)=A(\y)$, but it could also be something more complex, such as the boundary of a domain, as discussed in references~\cite{castrillon-candas2016,harbrecht2016,hiptmair2018}.
In the works mentioned above, the size of $\mathcal{O}_j$ depends then on the decay of $\lVert\psi_j\rVert_X$ as $j\rightarrow\infty$.
In particular, they do not take into account the support of $\psi_j$, $j\geq 1$.
The proofs presented here are based on what in~\cite{dung2022} is referred to as ``bootstrap'' arguments, that is, iterated differentiation, which, for the application to parametric domains, are in the spirit of the proofs in~\cite{harbrecht2016}, although using different estimates.

Results in the same spirit, although requiring quite different proof techniques, have been obtained in the lognormal case.
In that case, $A(\y)$ in~\eqref{eq:pde} is the exponential of a Gaussian random field, and the parameter $\y$, taking values in the whole $\mathbb{R}^{\mathbb{N}}$ with underlying Gaussian measure, corresponds to the image of the Gaussian random variables in the field expansion.
For more details, see, for instance,~\cite{ernst2018,hoang2014}.

In the last years, starting from~\cite{bachmayr2017a}, it has become clear that using \textsl{pointwise} instead of norm-wise bounds on the decay of $\psi_j$ as $j\rightarrow\infty$ can lead to a faster decay of the Taylor or Legendre coefficients of the PDE solution when the supports of $\psi_j$, $j\geq 1$, are localized.
The main idea is that, in this case, using norms to measure the effect of each parameter coordinate on the output's variability might be too pessimistic.
One should exploit instead that, at a specific location in the domain $D$, only a few functions contribute significantly to changes of the input $\mathfrak{p}$ (and so, in some way, to changes in the PDE solution).
The work~\cite{bachmayr2017a} considers the model problem~\eqref{eq:pde} and an \textsl{affine} dependence of the diffusion coefficient on the parameter (and fixed right-hand side).
Similar results have been shown in the lognormal case~\cite{bachmayr2017,dung2022}.
For a more general class of PDEs, pointwise bounds are also considered in~\cite{zech2018}, on which we comment at the end of Section~\ref{sec:summability-of-taylor-coefficients}.

Furthermore, the localized expansions, which these pointwise bounds exploit, can be beneficial in the computation of the polynomial expansions.
For example, computational domains can be meshed adaptively, depending on the support of the basis expansion.
An example in the case of stochastic Galerkin has been explored in~\cite{bachmayr2025} for the linear and lognormal coefficients in the diffusion problem~\eqref{eq:parameterized_poisson_pde}.

The results in this chapter build upon~\cite{bachmayr2017,bachmayr2017a}.
The main contribution is to extend the results in~\cite{bachmayr2017a}, which are for the affine case, to the case that the coefficient and right-hand side in~\eqref{eq:pde} depend \textsl{nonlinearly}, but smoothly, on the parameter.
The next chapter will show how this can be applied to treat elliptic problems on parametric domains.