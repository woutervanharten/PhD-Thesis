In this chapter, we focus on the heterogeneous parameterized Poisson equation~\eqref{eq:parameterized_poisson_pde}, which we repeat here for the reader's convenience:
\begin{equation}
    \begin{cases}
        -\nabla\cdot\left(A(\y)\nabla u(\y) \right) = F(\y), & \text{in } D,\\
        u(\y) = 0, & \text{on } \partial D,\\
        \text{for every }\bm{y} \in Y \coloneqq [-1, 1]^\mathbb{N}.
    \end{cases} \label{eq:pde}
\end{equation}
The parameter dependence of the input, entering~\eqref{eq:pde} via the PDE coefficient and right-hand side, propagates to the solution $u$, with $u(\y)\in H_0^1(D)$ for every $\y\in Y$, and to possible output functionals depending on it.
We are interested in constructing an efficient surrogate, or an approximant, for the parameter-to-solution map $\y \mapsto u(\y)$.
This surrogate is useful for both forward and inverse UQ.
Moreover, uses are not restricted to UQ as surrogates can be helpful in, for example, optimization and inference.
We focus on sparse \textsl{polynomial} surrogates, and later in Chapter~\ref{ch:locality-in-parameterized-domains}, we highlight potential implications for approximation properties of other surrogates and quadrature.

This chapter establishes a general framework for Chapter~\ref{ch:locality-in-parameterized-domains}, where we address the application to elliptic PDEs on \textsl{parameter-dependent domains} with fixed coefficients and right-hand side.
As we have seen in Chapter~\ref{ch:uncertainty-quantification-for-paramterized-domains}, the latter can be recast in the setting~\eqref{eq:pde}.
Motivated by the application to parametric domains, among many others, we are particularly interested in the case that $A$ and $F$ in~\eqref{eq:pde} depend smoothly but possibly in a highly nonlinear way on the parameter $\y$ in the present chapter.

We will start with a literature review in Section~\ref{sec:state-of-the-art}.
This is followed by introducing Taylor approximations to the parameter-to-solution map for problem~\eqref{eq:pde} in Section~\ref{sec:summability-of-taylor-coefficients}.
In the remainder of the latter, we prove the $\ell^p$-summability of the Taylor coefficients via a weighted $\ell^2$-summability argument.

Throughout the chapter, we use the following conventions.
We denote the spectral norm of a $n$-tensor by $\left|\,\cdot\,\right|_{n,2}$, and, for the special case $n=1$, we use the short notation $\left|\,\cdot\,\right|$ as used in the previous chapters.
For parameter- and space-dependent quantities, when using the aforementioned norms in proofs, we will omit explicit dependence on the space coordinate to lighten the notation.
For instance, for the diffusion coefficient in~\eqref{eq:pde}, $\left|A(\y)\right|_{2,2}$ will denote the spectral norm of the matrix $A(\y,\x)$, for given $\y\in Y$ and $\x\in D$.
For the norm on an infinite-dimensional Banach space $X$, instead, we use $\lVert\,\cdot\,\rVert_X$.